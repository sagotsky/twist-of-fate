% learning tex
% Twist of Fate (working title) docs

\documentclass[twocolumn]{report}
\usepackage{fullpage}		% wider page than default
\usepackage{wrapfig}		% wrap text around a figure
\usepackage{pifont}
%\pagestyle{pagenumbers}

\title{Twist of Fate}
\author{Jon Sagotsky}
%\date{October 2009}

\begin{document}
\maketitle
\tableofcontents

\chapter{The Basics}
\section{Core Mechanic - Bidding}

Twist of Fate is a diceless roleplaying game.  It retains the element of chance, though not randomness, by way of its core mechanic - bidding.  Each player has a number of skills and attributes as well as a pool of will power points.  When it comes	time to see if a character succeeds or fails, players spend a number of points, which is added to the skill or attribute score. 

This mechanic helps Twist of Fate focus on the collaborative storytelling side of roleplaying, without detracting from the game side.  The idea is that while characters have set skills, what they care about achieving should have an effect on the world around them.  This helps ensure that a character succeeds where he's supposed to, rather than happening to roll a criticalfailure or a botch.  This in turn goes to reinforce that a character's role in the story is the role the player wants to play.

\newpage
\section{Attributes}
To model a character's potential ability and talent, each character has a set of attributes.  These attributes range from 0-8 points, where 0 is entirely nonfunctional and 8 is nearly superhuman.  Unlike other RPGs, MoMa aims to work best for characters who have a balanced set of attributes instead of an extremely specialized set.

\begin{description}
\item[Strength]
Physical brawn.  This attribute is common for athletic and combative characters.

\item[Agility]
Dexterity and finesse for characters who are physically competent without relying on brute force.

\item[Brains]
Mental acumen.  A character with brains will have book smarts and good reasoning ability but fall victim to frequent zombie attacks. \textit{``Braaaains!''}

\item[Social]
Charisma and people skills.  This stat indicates how well your character interacts with others.

\item[Health]
Your ability to not die.  Unlike the other attributes, health is passive.  It does not make you more capable or effective, but keeps you on your feet long enough to use your other attributes.

\item[Magic (Optional)]
This attribute is optional and may not be present in all games.  Spellcasting is elemental, with each of the four active elements corresponding to earth, fire, water, or wind.

\end{description}

\newpage
\section{Skills}
While attributes measure a character's natural abilities, skills represent those abilities and techniques a character has practiced and learned.  Each skill is associated with an attribute.  Just like attributes, skills range from 0 to 8 points, 0 being untrained, 8 being among the best in the world. 

To use a skill, add your skill score to its relevant attribute's score.  If the result matches or exceeds the difficulty number (DN) assigned by your gamemaster, the skill succeeds.  Sometimes your skill will be used in direct opposition to another character in the game.  In these cases, instead of using a DN you make an opposed skill check and the player with the highest result wins.  

Simple, right?  Also boring.  Checking if attribute plus skill is greater than a difficulty number is about as simple as skill checks can get.  It's efficient but uninteresting.  Most of your skill checks (at least the ones that your character is interested in) will be more mechanically interesting than this.  Characters have a pool of willpower points.  These can be spent during skill checks to increase your score for that check.  They will be discussed in greater detail in the next section.

\newpage
\section{Willpower}

\subsection*{The Why of Willpower} 
Some gamers find the idea of resolving conflicts using willpower instead of dice to be questionable.  Others find it downright offensive.  What follows is an explanation of the thought process involved in coming up with MoMa.

Many moons ago, the author of this game was involved in a game of Deadlands.  Deadlands had a mechanic called fate chips.  Chips could be used to increase die rolls or escape bodily harm.  If you had any fate chips left by the end of the session you could spend them to increase your abilities.  Quite a few combats involved GM and PCs trading stacks of chips as one side powered up their attacks and the other evaded injury.  Eventually both sides ran out of chips combatants could die. 

I borrowed this idea when I first started GMing Dungeons and Dragons.  There is no need to enumerate the myriad mechanics used in emulating fate chips in D\&D.  The result was that chips could be spent to let players fudge the dice in their character's favor, within the bounds of the character.  Knights wouldn't fall off horses.  Sly tricksters wouldn't get tongue tied.  Essentially, \emph{fate chips allowed players to play the character they imagined, not the one the dice gave them.}

Twist of Fate seeks to provide mechanics that expand on this notion of giving narrative control back to players.  In this sense, MoMa prioritizes collaborative storytelling over gaming or simulation.

The willpower mechanics also serve to push the focus of the game toward interesting conflicts.  For instance, look at characters trying to climb a tree.  Checking if attribute plus skill is greater than the DN of a given tree is boring.  You can skim right over it.  Either the players succeed or they don't.  There is no reason to burden this part of the game with tedious skills - it has no effect on the narrative.  If the players really want to succeed here, they have the power to do so by spending willpower points.  

Willpower is a simple mechanic.  Each character has a number of willpower points.  They can be spent during skill checks after totalling your skill, but before resolving if you succeed.  In the case of opposed skill checks, players should simultaneously reveal how many willpower points are being spent.  Points are always spent, even on a failed check.

To make things more interesting, characters have separate pools of willpower bound to each attribute: strength, agility, brains, social, and (optionally) magic.  The availability of these points is determined by their attribute.  They represent characters having a little extra oomph in the area of their expertise.

Each attribute can hold a maximum of 5 \ding{53} points.  Each night of rest restores 3 \ding{53} points.  Each meal (up to three per day) restores 1 \ding{53} points.  So a character with 4 strength can have up to 20 strength points at one time, regains 12 strength points each night of rest, and 4 points at each meal.  These numbers are chosen to encourage players to spend their points instead of hoarding them.  It is expected that most encounters will require the expenditure of willpower points.

It doesn't take a lot of math ability to realize that an average strength character with no weapons training can spend his 20 strength points to best a brawny master with 8 strength and 8 weapon skill.  For this reason, the number of willpower points spent in one check is limited to the skill being used.  This means that at best you can double your skill in any one check.  Therefore the highest result of any check is 24 (before perks and other shenanigans).

\emph{
Because points are always spent, even on a failed check, you can end up with interesting game play even on the simplest of checks.  Let's say you're trying to climb a tree to get over a wall.  The tree has few limbs, so it's difficulty 10 to climb.  You have 4 points agility and 2 points of athletics, for a score of 6.  You try to climb at a 6 and fail.  Retry, spending a point for 7, and you still fail.  For some odd reason probably stemming to your childhood, you really want to get over this tree.  How many points to spend though?  2, then 3, and then 4 will eventually get you over but you'll have spent 10 willpower points where 4 were needed.  Skipping to a higher number could work, but it can also be wasteful.  A simple algorithm would be to only use even or odd numbers.  In this case, only odd would have you spend 1, 3, and 5 points for a total of 9.  Only even would be 2 and 4 for 6 points.  On the other hand if you were severely unathletic and 7 points were needed, odd would cost 1, 3, 5, and 7 points - total 16 as opposed to 2, 4, 6, and 8 for 20 points.  Now consider that you may not always have several rounds spent try to climb for cheap.  Or that going over the DN may have bonus effects.  If you were climbing the tree to escape an angry mob, wouldn't you want to climb it on your first try and twice as quickly as usual?  
}

overflow skills

Finally, there are a number of extra effects where you can spend willpower during combat.  Unlike the rest of the game these effects do not require a skill check.  More details are available in the combat chapter.
\chapter{Skills} 
To represent the initial learning curve in picking up a new skill, before any points can be invested in the skill a specialization must be purchased.  This is reversed from how specializations usually work in RPGs, in that you buy as many points in a skill as are economical and then pick up the specialty to go above and beyond those points.  Instead in this system the specialization is representative of the character's starting point that serves as the basis for the rest of the skill.  Specializations are listed after each skill name, in parentheses.

\section{Strength Skills}
Athletics (Climb, Swim, Throw, Lift)
Weapon (Bladed, polearm, bludgeon, two handed, gun, bow, fist)
Brawl/Grapple (hand to hand.  specialty for punching or grappling)
Carry (does carry in stones work?  are there specializations for carry?)
shield (important enoug that it gets its own.  how would shield specialties work?)

\section{Agility Skills}
Acrobatics (Tumble, Balance, jump, run)
Dodge
Subterfuge (hide, silence, trackless step)
blue collar?
vehicle (horse, car, boat, tank)
illicit (pickpocket, pick lock, holdout)
drive (ride/pilot/drive)
ranged weapons (is thrown a spec or separate?)

\section{Brains Skills}
Medicine (first aid, surgery, anatomy, pharmacy)
Science (Alchemy, Computers, whatever else fits)
Academics (any school subject.  potential for further specialization)
Knowledge (occult, legal, etc) (how is this different from academics)
Language (obviously this one will be a big abstraction.  maybe making the skill into a general linguistics thing but each specialty buys a language?)
Survival (tracking, camping, hunting)
white collar

\section{Social Skills}
Act (Character, Posture, Stage, Oratory)
Manipulate (Lie, pressure, manipulate, convince, pry, command, evade, intimidate, dip (compromise instead?) story?) (split this skill.  deception and straightforwardness maybe?)
Meet (Befriend, Chat, Gossip, Flirt, black market, connections?)
Scrutinize (Detect lies, observe)
where does etiquette go? (culture?)
culture 
lead

(linguistics is a special case.  language skill buys scholastic knowledge.  culture buys folk knowledge, letting people communicate as though they were locals.  culture can have other values too.)


\chapter{Flaws and Perks}

Characters are more than their attributes and skills.  They have loves and fears.  Flaws and perks serve to make characters more interesting by giving them more flavor than their ability stats.

\chapter{Combat}

fight!
\section{Combat Effects}
Additional combat effects were mentioned in the willpower chapter.  These are quick actions that take place on your combat round in addition to your combat action.  The cost of these effects is cumulative - the first costs 1 willpower, the 2nd costs 2, third 3, etc.  The only limit is how much willpower your character brings to the table.  Each effect corresponds to one of your attributes and can be bought with the appropriate willpower points.  They can be purchased before or after your combat action.

\begin{description}
\item[Hulk Up - Strength]
Each hulk up adds 1 to your melee combat check for next round.  After the first point, this becomes less and less economical but allows you to go and beyond what other characters are capable of in melee, as it does not count toward the point limit on your attack.  Hulk up is considered weaker than the other combat effects, but that's to make up for Strength being the melee attribute.

\item[Step - Agility]
Step allows you to move 1 square.  Ordinarily you can move or attack during your turn - this option lets you move to your opponent and attack in the same turn.

\item[Know Thine Enemy - Brains]
Choose another combatant and any skill or attribute to learn its value.  Continued uses of Know Thine Enemy will show you your enemy's strengths and weaknesses.  Note that some abilities may allow someone to mask their skills and attributes.

\item[Order - Social]
Giving orders to your teammates allows them to use combat effects during your turn.  Note that teammate actions will count towards point costs.  So on player A's turn, he can spend 1 social point to Order Player B to Step.  Player B then spends 2 agility points to move 1 square.  Play returns to A, who can spend 3 points on another combat effect.

\end{description}



\chapter{Advancement}

Save this for last.  Need to balance everything else before costs can be assigned.

\end{document}
